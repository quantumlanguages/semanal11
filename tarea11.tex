% type
\documentclass{article}

% page format
\usepackage[letterpaper, margin=1.25cm]{geometry}

% math env
\usepackage{amsmath}

% math symbs
\usepackage{amssymb}

% inferences
\usepackage{ebproof}

% macros
\newcommand{\lm}{\lambda}
\newcommand{\G}{\Gamma}
\newcommand{\x}{\times}
\newcommand{\bx}{\rightarrow^{\star}}
\newcommand{\tx}[1]{\texttt{#1}}
\newcommand{\ra}{\rightarrow}

\title{
    Lenguajes de Programación 2020-1\\
    Facultad de Ciencias UNAM\\
    Ejericio Semanal 11
}

\author{
    Sandra del Mar Soto Corderi\\
    Edgar Quiroz Castañeda
}

\date{
    7 de noviembre de 2019
}

\begin{document}
    %header
    \maketitle
	\begin{enumerate}
		\item {
		Considera el siguiente programa

		\tx{exception Excpt of int ;}\\
		\tx{fun twice (f , x ) = handle f ( f ( x ) ) with Excpt ( x ) = > x ;}\\
		\tx{fun pred ( x ) = if iszero x then raise Excpt ( x ) else x - 1;}\\
		\tx{fun smart ( x ) = handle 1 + pred ( x ) with Excpt ( x ) = > 1;}
		
		Evalúa las siguientes expresiones utilizando la semántica operacional de
		la máquina K extendida para manejar excepciones con valor. (pueden
		obviarse algunos pasos pero no los que involucran manejo de excepciones.
		En cada caso describe qué excepción fue lanzada y en dónde.)

		\begin{itemize}
			\item {
			\tx{twice(smart, 0)}

			Cuando se tenga una aplicación con valores bloqueados, se saltarán 
			los pasos hasta el cálculo que los concierna, ignorando los pasos 
			intermedios para verificar que son valores.

			\begin{align*}
				&\square \succ \tx{app(twice, (smart, 0))} \\
				&\ra_k \square \succ \tx{twice[f := twice, x := 0]}\\
				&\ra_{\beta} \square \succ \tx{handle(app(smart, app(smart, 0)), x.x)}\\
				&\ra_k \tx{handle(\_, x.x)} \succ \tx{app(smart, app(smart, 0))} \\
				& \ra_k \tx{app(smart, \_)}, \tx{handle(\_, x.x)} \succ \tx{app(smart, 0)}\\
				&\ra_k \tx{app(smart, \_)}, \tx{handle(\_, x.x)} \succ \tx{smart[x := 0]} \\
				&\ra_{\beta} \tx{app(smart, \_)}, \tx{handle(\_, x.x)} \succ \tx{handle(sum(1, app(pred,0)), x.1)} \\
				&\ra_k \tx{handle(\_, x.1)}, \tx{app(smart, \_)}, \tx{handle(\_, x.x)} \succ \tx{sum(1, app(pred,0))} \\
				&\ra_k \tx{sum(1, \_)}, \tx{handle(\_, x.1)}, \tx{app(smart, \_)}, \tx{handle(\_, x.x)} \succ \tx{app(pred,0)} \\
				&\ra_k \tx{sum(1, \_)}, \tx{handle(\_, x.1)}, \tx{app(smart, \_)}, \tx{handle(\_, x.x)} \succ \tx{pred[x:=0]} \\
				&\ra_{\beta} \tx{sum(1, \_)}, \tx{handle(\_, x.1)}, \tx{app(smart, \_)}, \tx{handle(\_, x.x)} \succ \tx{if(iszero(0), raise(0), dif(0, 1))}\\
				&\ra_k \tx{if(\_, raise(0), dif(0, 1))}, \tx{sum(1, \_)}, \tx{handle(\_, x.1)}, \tx{app(smart, \_)}, \tx{handle(\_, x.x)} \succ \tx{iszero(0)} \\
				&\ra_k \tx{if(\_, raise(0), dif(0, 1))}, \tx{sum(1, \_)}, \tx{handle(\_, x.1)}, \tx{app(smart, \_)}, \tx{handle(\_, x.x)} \prec \tx{true} \\
				&\ra_k \tx{sum(1, \_)}, \tx{handle(\_, x.1)}, \tx{app(smart, \_)}, \tx{handle(\_, x.x)} \ll \tx{raise(0)} \\
			\end{align*}

			En este punto se lanza una excepción de tipo \tx{Excpt} con un valor
			de 0.

			\begin{align*}
				&\ra_k \tx{handle(\_, x.1)}, \tx{app(smart, \_)}, \tx{handle(\_, x.x)} \ll \tx{raise(0)} \\
				&\ra_k \tx{app(smart, \_)}, \tx{handle(\_, x.x)} \succ \tx{1[x:=0]} \\
				&\ra_{\beta} \tx{app(smart, \_)}, \tx{handle(\_, x.x)} \succ 1 \\
				&\ra_k \tx{handle(\_, x.x)} \succ \tx{smart[x:=1]} \\
				&\ra_{\beta} \tx{handle(\_, x.x)} \succ \tx{handle(sum(1, app(pred,1)), x.1)} \\
				&\ra_k \tx{handle(\_, x.1)}, \tx{handle(\_, x.x)} \succ \tx{sum(1, app(pred,1))}\\
				&\ra_k \tx{sum(1, \_)}, \tx{handle(\_, x.1)}, \tx{handle(\_, x.x)} \succ \tx{app(pred,1)}\\
				&\ra_k \tx{sum(1, \_)}, \tx{handle(\_, x.1)}, \tx{handle(\_, x.x)} \succ \tx{pred[x:=1]}\\
				&\ra_{\beta} \tx{sum(1, \_)}, \tx{handle(\_, x.1)}, \tx{handle(\_, x.x)} \succ \tx{if(iszero(1), raise(1), dif(1, 1))}\\
				&\ra_k \tx{if(\_, raise(1), dif(1, 1))}, \tx{sum(1, \_)}, \tx{handle(\_, x.1)}, \tx{handle(\_, x.x)} \succ \tx{iszero(1)}\\
				&\ra_k \tx{if(\_, raise(1), dif(1, 1))}, \tx{sum(1, \_)}, \tx{handle(\_, x.1)}, \tx{handle(\_, x.x)} \prec \tx{false}\\
				&\ra_k \tx{sum(1, \_)}, \tx{handle(\_, x.1)}, \tx{handle(\_, x.x)} \succ \tx{dif(1, 1)}\\
				&\ra_k \tx{sum(1, \_)}, \tx{handle(\_, x.1)}, \tx{handle(\_, x.x)} \prec 0\\
				&\ra_k \tx{handle(\_, x.1)}, \tx{handle(\_, x.x)} \succ \tx{sum(1, 0)}\\
				&\ra_k \tx{handle(\_, x.1)}, \tx{handle(\_, x.x)} \prec 1\\
				&\ra_k \tx{handle(\_, x.x)} \prec 1\\
				&\ra_k \square \prec 1\\
			\end{align*}
			}
			\item {
			\tx{twice(pred, 1)}

			Cuando se tenga una aplicación con valores bloqueados, se saltarán 
			los pasos hasta el cálculo que los concierna, ignorando los pasos 
			intermedios para verificar que son valores.

			\begin{align*}
				&\square \succ \tx{app(twice, (pred, 1))} \\
				&\ra_k \square \succ \tx{twice[f := pred, n := 1]}\\
				&\ra_{\beta} \square \succ \tx{handle(app(pred, app(pred, 1)), x.x)} \\
				&\ra_k \tx{handle(\_, x.x)} \succ \tx{app(pred, app(pred, 1))} \\
				&\ra_k \tx{app(pred, \_)}, \tx{handle(\_, x.x)} \succ \tx{app(pred, 1)} \\
				&\ra_k \tx{app(pred, \_)}, \tx{handle(\_, x.x)} \succ \tx{pred[x:=1]} \\
				&\ra_{\beta} \tx{app(pred, \_)}, \tx{handle(\_, x.x)} \succ \tx{if(iszero(1), raise(1), dif(1, 1))} \\
				&\ra_k \tx{if(\_, raise(1), dif(1, 1))}, \tx{app(pred, \_)}, \tx{handle(\_, x.x)} \succ \tx{iszero(1)} \\
				&\ra_k \tx{if(\_, raise(1), dif(1, 1))}, \tx{app(pred, \_)}, \tx{handle(\_, x.x)} \prec \tx{false} \\
				&\ra_k \tx{if(\_, raise(1), dif(1, 1))}, \tx{app(pred, \_)}, \tx{handle(\_, x.x)} \succ \tx{iszero(1)} \\
				&\ra_k \tx{app(pred, \_)}, \tx{handle(\_, x.x)} \succ \tx{dif(1, 1)} \\
				&\ra_k \tx{app(pred, \_)}, \tx{handle(\_, x.x)} \prec 0 \\
				&\ra_k \tx{handle(\_, x.x)} \succ \tx{app(pred, 0)} \\
				&\ra_k \tx{handle(\_, x.x)} \succ \tx{pred[x:=0]} \\
				&\ra_{\beta} \tx{handle(\_, x.x)} \succ \tx{if(iszero(0), raise(0), dif(0, 1))} \\
				&\ra_k \tx{if(\_, raise(0), dif(0, 1))}, \tx{handle(\_, x.x)} \succ \tx{iszero(0)} \\
				&\ra_k \tx{if(\_, raise(0), dif(0, 1))}, \tx{handle(\_, x.x)} \prec \tx{true} \\
				&\ra_k \tx{handle(\_, x.x)} \ll \tx{raise(0)} \\
			\end{align*}

			En este punto se lanza una excepción de tipo \tx{Excpt} con un valor
			0.

			Se continua la evaluación del programa.

			\begin{align*}
				&\ra_k \square \succ \tx{x[x:=0]} \\
				&\ra_{\beta} \square \succ 0 \\
				&\ra_k \square \prec 0 \\
			\end{align*}
			}
		\end{itemize}
		
		}
	\end{enumerate}
\end{document}